The advance of microprocessing and communication technologies have provided a major impetus to mobile services. Mobile services have emphasized much more on the contents of applications and service experiences than before, as early mobile services only tried to keep the mobility of services. On the other hand, not only apparatus, office and production devices are being equipped with powerful processors and intercommunication modules, but diverse sensors are deployed in environments, which implies that the ubiquitous age is coming. The synergy of ambient intelligence and wide area mobility of mobile networks will improve the communication of people, bring more informations and services experiences since the more it is able to gather from the environments.

However, the vision is challanged by many technology issues: along with increasing of wireless access scheme, most of users will be served by several access networks, thus, technologies, e.g. multi-mode terminal, software defined radio, and cognitive radio, are being developed. On the other hand, the proliferation of service contents and context are challanging the autonomous configuration and management, especially when the user is moving around. Although technologies such as smart home, smart office, have been developed for several years, the smart space around a user is hard to configuration and management since the members join or leave frequently, which is addressed by the dissertation.

Firstly, the service environment is considered under the condition of synergy of mobile and ubiquitous. With the abstraction of service capability of ambient intelligent devices, applications are able to be built with standardard building blocks, which enables the mobility of mobile ubuquitous services. Moreover, the cooperation of devices attaching to different access networks enables another type of ``reconfigurable terminal'', which is named as ``aggregating reconfigurable distributed terminal''.

The characteristics of mobile personal networks, which are consist of ambient intelligent devices, are analyzed at first, and mobility model of mobile personal networks are addressed as well. Then a suit of service discovery methods specifically for mobile personal networks are proposed, which emphase on keeping system stable when a mass of devices are joining or leaving. And the decentrialized directory is also provided, which is working as the infrastructure of the autonomous configuration and management mechanisms.

At last, a brief conclusion is drawn and the issues for further study are addressed.
